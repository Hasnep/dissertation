%% preamble

%% font encoding
\usepackage[utf8]{inputenc}
\usepackage[T1]{fontenc}

%% language settings
\usepackage[UKenglish]{babel}

%% Line spacing
\usepackage{setspace}
\setstretch{1.25}

%% Quotes
\usepackage{csquotes}
\newcommand{\qt}[1]{\enquote{#1}}
\let\qtc=\blockcquote

%% Cut command
\newcommand*{\cut}{[\ldots{}]}

%% Maths packages
\usepackage{amsmath}
\usepackage{amssymb}

%% Nice fraction command
\newcommand{\nicefrac}[2]{{#1}/{#2}}

%% Matrix transpose notation
\newcommand*{\transpose}{^\mathsf{T}}

%% Gaussian Process macro
\newcommand*{\gp}{\mathcal{GP}}

%% Redefine vector macro
\usepackage{bm}
\let\oldvec=\vec
\let\vec=\bm
% \renewcommand*{\vec}{\bm}

%% Bibliography
\usepackage[ %
	backend=biber,
	sorting=nyt, % sort by name, year, title
	style=authoryear, % use (author, year) style
	maxbibnames=10, % maximum number of names to include in a bibliography entry
	minbibnames=10,
	maxcitenames=3, % maximum number of names to include in a citation
	mincitenames=1,
	giveninits=true, %
	uniquelist=false, % never disambiguate names
	uniquename=false, % never disambiguate names
	date=year % remove months from bibliography
]{biblatex}
\addbibresource{../bibliography/dissertation.bib}
\addbibresource{../bibliography/literaturereview.bib}
\DeclareNameAlias{sortname}{family-given} % bibliography show names as surname, then first name
\renewbibmacro{in:}{} % remove the "in: " before a journal's name
\renewcommand{\UrlFont}{\small\tt} % make URLs appear in a smaller font
\usepackage{xpatch}
\xpatchbibmacro{name:andothers}{\bibstring{andothers}}{\bibstring[\emph]{andothers}}{}{} % make et al. in italics
\SetCiteCommand{\parencite} % Set quotation citation style for csquotes

%% URLs
\usepackage[hidelinks]{hyperref}

%% Set notation
\newcommand*{\reals}{\mathbb{R}}

%% Distributions
\newcommand*{\distributed}{\sim}
%\DeclareMathOperator{\normdist}{N}
\newcommand*{\normdist}{\mathcal{N}}

%% Statistical functions
% \DeclareMathOperator{\e}{E}
% \DeclareMathOperator{\Var}{Var}
% \DeclareMathOperator{\cov}{Cov}

%% Derivatives
\usepackage{diffcoeff}

%% Acronyms
\usepackage{acronym}
% AI
\acrodef{AI}{artificial intelligence}
% AIC
\acrodef{AIC}{Akaike information criterion}
% ANN
\acrodef{ANN}{artificial neural network}
% CV
\acrodef{CV}{cross-validation}
% DL
\acrodef{DL}{deep learning}
% DNN
\acrodef{DNN}{deep neural network}
% GP
\acrodef{GP}{Gaussian process}
\acrodefplural{GP}{Gaussian processes}
% GPR
\acrodef{GPR}{Gaussian process regression}
% GPU
\acrodef{GPU}{graphics processing unit}
% ML
\acrodef{ML}{machine learning}
\acrodefindefinite{ML}{an}{a}
% ReLU
\acrodef{ReLU}{rectified linear unit}
% LASSO
\acrodef{LASSO}{least absolute shrinkage and selection operator}
% Latin acronyms
\newcommand{\eg}{e.g.}
\newcommand{\ie}{i.e.}

%% Set title properties
\title{Can statistics help us to understand deep learning?}
\author{Johannes Smit}
\date{May 2019}

%% Graphics
\usepackage{graphicx}
\graphicspath{{figures/}}
% SVGs
\usepackage[inkscape=overwrite, inkscapepath=svgsubdir]{svg}
% \usepackage[inkscapepath=svgsubdir]{svg}
\svgpath{{figures/}}
% default figure size
\newlength{\figwidth}
\setlength{\figwidth}{12cm}
% \usepackage{subcaption} % sub figures
% line colours
\newcommand{\traincolour}{black}
\newcommand{\truthcolour}{dark yellow}
\newcommand{\inordercolour}{light blue}
\newcommand{\reversedcolour}{green}
\newcommand{\shuffledcolour}{light yellow}
\newcommand{\anncolour}{orange}
\newcommand{\gpcolour}{pink}

%% Tables
\usepackage{booktabs}
\newcommand{\includetable}[1]{\input{tables/#1}}

%% Fix appendix titles
\usepackage[titletoc]{appendix}

%% Code
\usepackage{xcolor}
\usepackage{listings}
\usepackage{textcomp}
\definecolor{codegrey}{gray}{0.95}
\lstset{
	backgroundcolor=\color{codegrey},
	tabsize=2,
	language=R,
	basicstyle=\small\ttfamily\setstretch{0.9},
	upquote=true,
	columns=fullflexible ,
	showstringspaces=false,
	extendedchars=true,
	showtabs=false,
	showspaces=false,
	showstringspaces=false,
	keywordstyle=\color[rgb]{0,0,1},
	commentstyle=\color[rgb]{0.133,0.545,0.133},
	stringstyle=\color[rgb]{0.627,0.126,0.941}
}

%% Temporary
% to-do environment
\usepackage{framed}
\newenvironment{todo}{\begin{framed}\textbf{To-do:}~\itshape}{\end{framed}}
% note macro
\newcommand{\note}[1]{\textit{[#1]}}
% reword macro
\usepackage{soul}
\setulcolor{red}
\newcommand{\reword}[1]{\ul{#1}}
