% !TeX root = ../dissertation.tex

\chapter*{Abstract}
\addcontentsline{toc}{chapter}{Abstract}

Machine learning and deep neural networks have seen widespread success in many of modern life --- sometimes visibly, as with driverless cars, but in some cases more discreetly, such as the use of machine learning algorithms in the U.S. judicial system.
Due to their hierarchical structure, deep neural networks are a `black box' which no human can understand, which could cause problems when a machine learning algorithm does something unforeseen.
Statistical methods such as Gaussian processes may offer a way to look inside this black box, as they offer a similar flexibility and wide range of uses, and are much more easily interpreted by humans.
In this project, a simple non-linear function was used to train a deep neural network and then multiple regression and Gaussian processes were used to model the output of the neural network.
Regularisation methods such as LASSO were used to reduce the regression model to a more human understandable form, which was then used as the mean function of a Gaussian process to further improve the fit of the model.
