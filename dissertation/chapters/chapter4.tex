% !TeX root = ..\dissertation.tex

\chapter{Conclusion}

You can use these methods to:\\
Approximate the output of the neural network,\\
Put the output of the neural network in human understandable terms,\\
Find which terms are significant in the neural network model.\\

If you're trying to discover something about the real world, these techniques all rely on the neural network accurately reflecting reality which they probably don't.
The fit of the statistical method will only be as good as the fit of the neural network, these techniques cannot diagnose a bad neural network.

% One problem with using a \ac{GP} is that prediction with \(n\) training datapoints involves inverting an \(n \times n\) matrix, which makes it impractical for predicting many points quickly.
% On the other hand, neural networks take a long time to train, but once trained, predication only involves calculating the output of each neuron, so predictions can be made very quickly.
% This means that this technique is not useful for live situations where a neural network is being trained and an analysis is needed alongside the training.
% However, after a neural network is trained, a \ac{GP} can be fitted and predictions pre-calculated which can then be used live.

\section{How well have each of the attempts worked?}

The stepwise regression was not as good as the \ac{LASSO} technique, which worked better.
The \ac{GP} was good, as expected.

It is not clear how well these techniques would transfer to higher dimensions and more complex situations as the \ac{GP} took a long time to fit, and the regression didn't fit that well on a simple example.

\section{What could be improved?}

\section{How useful would more research on this topic be?}

\section{Future research}

These techniques rely on the fit of the neural network to recover the original information, and the neural network could definitely fit better.

If using this, it would be sensible to first perform sensitivity analysis to find the significant variables.

For more complex applications, the use of \acp{GP} could be extended to deep (also known as hierarchical) \acp{GP}, which are analogous to \acp{DNN} (see Section~\ref{sec:deep-learning}).
\reword{They involve} chaining the output of one \ac{GP} into another to better model nonlinearity~\autocite{damianou2013}.

One simple possibility to extend the models proposed would be to use multiple regression to find a trend and then use this as a mean function in a \ac{GP}.
This should allow the regression to capture most of the pattern and then the \ac{GP} will provide a flexible way to capture the residuals.

\section{Conclusion's conclusion}

Deep learning is fast becoming a \reword{ubiquitous tool} in many aspects of modern life --- sometimes clearly visible, as with driverless cars, but in some cases more discreetly, such as the use of machine learning algorithms in American courts.
As \acp{DNN} were designed to mimic a biological brain, a \reword{phenomenon} that is still not well understood, they are a \qt{black box} \reword{whose} reasoning is impossible to understand.
Statistical methods such as \acp{GP} may offer a way to look inside this black box, as they offer a similar flexibility and wide range of uses, and are much more easily interpreted by humans.
So far, much of the work that has been done involving \acp{GP} and machine learning has been comparative, rather than using one to model the other.
